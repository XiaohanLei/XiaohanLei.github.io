\vspace{-4pt}
\section{Method}
\label{sec:method}


\begin{figure*}[t]
  \centering
  \includegraphics[width=0.99\linewidth]{pic/main.pdf}
  \vspace{-8pt}
  \caption{Our proposed model architecture. The policy takes a history of $T_o$ raw 3D point clouds $\mathcal{P}_t = (\mathbf{P}_{t-T_o+1}, \dots, \mathbf{P}_t)$ and a language instruction $L$ as input. 
 \textit{Observation Tokenizer}: Each point cloud $\mathbf{P}_k$ in the history is processed via Farthest Point Sampling (FPS) and PointNets to extract local geometric tokens and a global scene context. The tokens from each time step are concatenated to form the final observation token sequence.
 Language is encoded by a T5 tokenizer~\cite{2020t5}. 
 \textit{Structural Action Tokenizer}: 
 Guided by the manipulator’s morphology, the Embodied Joint Codebook produces structural-centric embeddings aligned with the action dimension $D_a$, which are added to the time-stepped noisy tokens $\mathbf{A}_t^\tau$.
 \textit{Structural Action Transformer:} A
 DiT~\cite{peebles2023scalable} with causal masking predicts the action velocity
 field. This field is then integrated via an ODE solver to produce the final
 action chunk $\mathbf{A}_t$.}
  \vspace{-16pt}
  \label{fig:main}
\end{figure*}

% We detail our approach for cross-embodiment dexterous manipulation from 3D observations. We first formulate the problem in Section 3.1. We then introduce our core model as a conditional continuous-time normalizing flow in Section 3.2, detailing its objective. Finally, we describe the network architecture in Section 3.3, focusing on our novel structural action representation and the embodied joint codebook.

\subsection{Problem Formulation}

We formulate the control problem for
dexterous hands as a conditional generative modeling task. At each time step
$t$, the policy receives an observation $o_t$ which consists of a history of the last $T_o$ raw 3D point clouds, $\mathcal{P}_t = (\mathbf{P}_{t-T_o+1}, \dots, \mathbf{P}_t)$, where each $\mathbf{P}_k \in \mathbb{R}^{N \times 3}$ ($N$ is the
number of points), and a natural language instruction $L$. The goal is to learn a policy $\pi$ that
predicts a chunk of future actions $\mathbf{A}_t$. 
As motivated in our
introduction, we challenge the conventional temporal-centric action
representation $(T, D_a)$. Instead, we define the action chunk from a structural-centric
perspective as $\mathbf{A}_t \in \mathbb{R}^{D_a \times T}$, where $D_a$ is the
dimensionality of the robot's action space and $T$ is the prediction horizon.
Each row $j_i \in \mathbb{R}^{T}$ of $\mathbf{A}_t$ represents the entire
future trajectory for the $i$-th joint. The policy $\pi$ thus models the
conditional distribution 
% $p(\mathbf{A}_t | o_t) = p(\mathbf{A}_t |
% \mathcal{P}_t, L)$. 
$p(\mathbf{A}_t | o_t)$. 
During closed-loop control, this action chunk $\mathbf{A}_t$
is generated in a receding-horizon manner. A subset of the predicted actions is
executed, and the policy is re-queried with the new observation, enabling
continuous feedback and correction.

% \vspace{-2pt}
\subsection{Policy as a Conditional Normalizing Flow}
% \vspace{-2pt}

We model the complex, high-dimensional conditional distribution $p(\mathbf{A}_t | o_t)$ using a continuous-time normalizing flow (CNF)~\cite{lipman2022flow}. We frame this as learning a conditional velocity field $v(\mathbf{A}_t^\tau, \tau, o_t)$ that transports a standard Gaussian $\mathcal{N}(0, I)$ to the action distribution. We parameterize the velocity field with a neural network $\epsilon_\theta(\mathbf{A}_t^\tau, \tau, o_t)$, where $\tau \in [0, 1]$ is the flow time. The network is trained to minimize the following objective:
\begingroup
\setlength{\abovedisplayskip}{-0pt}
\setlength{\belowdisplayskip}{2pt}
\begin{equation}\label{eq:fm_objective}
\resizebox{\linewidth}{!}{$
\mathcal{L}(\theta)=\mathbb{E}_{\tau\sim\mathcal{U}(0,1),\mathbf{A}_t^0\sim\mathcal{N}(0,I),\mathbf{A}_t^1\sim\mathcal{D}}
\left[\left\|\epsilon_\theta(\mathbf{A}_t^\tau,\tau,o_t)-(\mathbf{A}_t^1-\mathbf{A}_t^0)\right\|^2\right],
$}
\end{equation}
\endgroup
where $\mathcal{D}$ is the dataset of ground-truth action chunks, $\mathbf{A}_t^1 \sim \mathcal{D}$ is a ground-truth action, $\mathbf{A}_t^0 \sim \mathcal{N}(0,I)$ is a noise sample, and $\mathbf{A}_t^\tau = (1-\tau)\mathbf{A}_t^0 + \tau \mathbf{A}_t^1$. The conditioning observation $o_t$ is the one associated with the target action $\mathbf{A}_t^1$.

At inference time, we sample a noise $\mathbf{A}_t^0 \sim \mathcal{N}(0,I)$ and generate the final action chunk $\mathbf{A}_t^1$ by solving the ordinary differential equation (ODE) $\frac{d\mathbf{A}_t^\tau}{d\tau} = \epsilon_\theta(\mathbf{A}_t^\tau, \tau, o_t)$ from $\tau=0$ to $\tau=1$. This is done with a numerical solver, such as a single Euler integration step, which recovers the probability flow in one function evaluation (1-NFE).

% \vspace{-2pt}
\subsection{Policy Network Architecture}
% \vspace{-2pt}

As shown in Figure~\ref{fig:main}, we design our velocity field model
$\epsilon_{\theta}$ as a Transformer~\cite{vaswani2017attention} architected to
ingest our novel structural action tokens and the multi-modal conditioning
observation tokens. The architecture is divided into three components: an
\textbf{Observation Tokenizer} that tokenizes the historical 3D point clouds
and language inputs, 
a \textbf{Structural Action Tokenizer} that embed structural priors into action tokens,
and a \textbf{Structural Action Transformer} that predicts the
action velocity field conditioned on the observations.

\vspace{-2pt}
\subsubsection{Observation Tokenizer}
\vspace{-2pt}

Our hierarchical point cloud tokenizer is designed to
capture both local geometric details and the global scene context from the entire observation history $\mathcal{P}_t$. We apply a shared encoding module to each of the $T_o$ point clouds $\mathbf{P}_k \in \mathcal{P}_t$. For a given $\mathbf{P}_k$, we first use
Farthest Point Sampling (FPS) to select $M$ local group centers $C_k = \{c_1,
..., c_M\} \subset \mathbf{P}_k$. For each center $c_{i}$, we query its $K$
nearest neighboring points to form a local group $\mathbf{P}_{k,i} \subset
\mathbf{P}_k$. Each local group $\mathbf{P}_{k,i}$ is processed by a shared
PointNet~\cite{qi2017pointnet} to extract a local geometric feature
$f_{k,i}\in\mathbb{R}^{d_{feat}/T_o}$. The 3D coordinates of the group centers $C_k$
are passed through an MLP to create positional embeddings
$p_{k,i}\in\mathbb{R}^{d_{feat}/T_o}$. These are combined
($f_{k,i}^{\prime}=f_{k,i}+p_{k,i}$) as $M$ local point tokens,
$tok_{l,k}\in\mathbb{R}^{M\times d_{feat}/T_o}$. To leverage the
permutation-invariant nature of these local patches, we apply a random shuffle
to the $tok_{l,k}$ sequence during training as data augmentation. In parallel,
to capture holistic scene understanding, we feed the entire raw point cloud
$\mathbf{P}_k$ into a separate PointNet-based encoder~\cite{qi2017pointnet}.
This module produces a single global token, $tok_{g,k} \in
\mathbb{R}^{d_{feat}/T_o}$, which represents the overall scene context for that timestep. This global
token does not receive any positional embedding.

This process yields $T_o$ global tokens and $T_o$ sets of $M$ local tokens. These are concatenated to form the full point cloud history sequence: 
\begingroup
\setlength{\abovedisplayskip}{1pt}
\setlength{\belowdisplayskip}{2pt}
\begin{equation}\label{eq:hist_token}
% \small
\begin{aligned}
tok&_{hist} = \text{Cat}\big(\text{Cat}(tok_{g,\,t-T_o+1},\dots,tok_{g,\,t}),\\
&\text{Cat}(tok_{l,\,t-T_o+1},\dots,tok_{l,\,t})\big)
\in\mathbb{R}^{(1+M)\times d_{feat}},
\end{aligned}
\end{equation}
\endgroup
where Cat means concatenation.
The language instruction $L$ is
tokenized and encoded using a pre-trained T5 encoder~\cite{2020t5} to produce a
sequence of language tokens $tok_{lang}\in\mathbb{R}^{L_{lang}\times
d_{feat}}$. The final observation tokens are formed by concatenating the
historical tokens and the language tokens: $tok_{obs} =
\text{Cat}(tok_{hist}, tok_{lang})$. This sequence serves
as the conditioning prefix for the model.

\vspace{-2pt}
\subsubsection{Structural Action Tokenizer}
\vspace{-2pt}

The noisy action $\mathbf{A}_{t}^\tau
\in\mathbb{R}^{D_{a}\times T}$ is treated as a sequence of $D_{a}$ tokens,
where each token represents the entire temporal trajectory for a single joint.
This high-dimensional temporal vector (dim $T$) contains significant
redundancy. We compress it by passing each of the $D_a$ joint trajectories
through a shared MLP, projecting it from $T$ dimensions to a lower-dimensional
embedding $d_{feat}$ (\textit{e.g.}, from $64$ to $16$). This results in an
embedded action sequence $tok_{act}\in \mathbb{R}^{D_{a}\times d_{feat}}$.

To resolve the ambiguity between the $D_a$
joint tokens and explicitly embed structural priors, we introduce an
\textbf{Embodied Joint Codebook}. This codebook is derived from the
manipulator's morphology. For any given hand, we define each joint $j$ as a
three-part triplet $J_{j} = (e, f, r)$:
\begin{itemize}
   \item $e\in\mathbb{Z}$: The \textbf{Embodiment ID}, a unique identifier
for the manipulator (\textit{e.g.}, ShadowHand, XHand).
   \item $f\in\mathbb{Z}$: The \textbf{Functional Category}. Inspired by
human hand anatomy, we classify joints by their functional role, such as
Carpometacarpal (CMC), Metacarpophalangeal (MCP), Proximal Interphalangeal
(PIP), or Distal Interphalangeal (DIP) joints.
   \item $r\in\mathbb{Z}$: The \textbf{Rotation Axis}, describing the
joint's primary motion, \textit{e.g.}, Flexion/Extension, Abduction/Adduction,
or Pronation/Supination.
\end{itemize}
The complete mapping table covering embodiment, function category and rotation axis used in our experiments is provided in Appendix.
Each element in the triplet $(e,f,r)$
indexes a separate learnable embedding table. The final codebook embedding $C_{j}\in\mathbb{R}^{d_{feat}}$ for
joint $j$, is the sum of its three component
embeddings. This design is central to handling heterogeneity: two different
hands (different $e$) may share the same functional joint (same $f$) and
rotation (same $r$), resulting in similar codebook embeddings that prime the
model for transfer learning. The final input sequence for the action
tokens is formed by adding the codebook embeddings to the compressed action
trajectories: $tok_{input\_act} = tok_{act} + \mathbf{E}$, where
$\mathbf{E} \in \mathbb{R}^{D_a \times d_{feat}}$ is the matrix of codebook
embeddings for the manipulator. 

\vspace{-2pt}
\subsubsection{Structural Action Transformer}
\vspace{-2pt}

The $tok_{input\_act}$ sequence is then
concatenated with the observation tokens $tok_{obs}$. This combined sequence
is fed into the DiT~\cite{peebles2023scalable}. 
We modify the DiT's
self-attention mask to enforce causal masking, where $tok_{obs}$ only
attends to other observation tokens, 
and $tok_{input\_act}$ attends to
all observation tokens and all other action tokens. 
The output tokens from the
DiT corresponding to the action sequence are passed through a final MLP to produce
the predicted action velocity field
$\epsilon_{\theta}(\mathbf{A}_{t}^{\tau},\tau,o_{t})$.


